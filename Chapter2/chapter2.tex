%!TEX root = ../thesis.tex
%*******************************************************************************
%*********************************** Second Chapter *****************************
%*******************************************************************************

\chapter{Background Research}\label{Background}  % Background

\ifpdf
    \graphicspath{{Chapter2/Figs/Raster/}{Chapter2/Figs/PDF/}{Chapter2/Figs/}}
\else
    \graphicspath{{Chapter2/Figs/Vector/}{Chapter2/Figs/}}
\fi
%********************************** % Water Distribution Systems  **************************************
\section{Water Distribution Systems}\label{waterdistributionsystems}
Water distribution systems are vital infrastructure networks required to deliver potable\footnote{Water suitable for drinking \citep{watereducationfoundation_2014}} water from centralised sources - such as reservoirs or water treatment plants - to customers. These systems can serve residential, commercial, and industrial areas, ensuring that water is available at required pressures while minimising cost. \citep{NAP11728}
\subsection{Structure and Components}
Typically a water distribution network consists of interconnected pipes and nodes - representing junctions, buildings and supply sources. Valves and storage tanks can also be included \citep{Christodoulou2018}. The design of such a system must account for hydraulic requirements, topological considerations - ensuring a connected network - and economic efficiency: minimising construction and operational costs such as piping length.
\subsection{Design Considerations}
Designs of these systems are inherently multidisciplinary. Careful considerations are required at all stages to ensure water:
\begin{enumerate}
    \item Reaches all locations at sufficient pressure
    \item Flows through pipes that can physically handle the required capacity
    \item Travels in directions that prevent stagnation and contamination
\end{enumerate}

Additionally, urban planning, environmental impact, and maintenance costs influence network configuration.
\subsection{Traditional Approaches}
Traditionally, water networks were designed using deterministic, rule-based methods. Engineers would manually lay out pipe routes based on heuristics and trial-and-error simulations. Hydraulic simulation tools enabled modelling of water pressure and flow but often required expert intervention for network layout. \citep{w11030562}
\subsubsection{Limitations}
Manual design methods can be subjective and, most importantly, time-consuming and suboptimal. This means that designing water distribution systems can take an incredibly long time when being done manually and there is no guarantee the best solution has been calculated. As networks grow in complexity and scale, manually ensuring all constraints are satisfied and the system is optimal becomes infeasible. \citep{Zhou2022}

\subsection{Optimisation-Based Techniques}\label{optimisationinwater}
Advances in computational power and mathematical modelling have enabled the application of optimisation techniques for water network design. These methods treat the network as a graph-based problem where nodes and edges represent physical elements, and constraints ensure hydraulic feasibility. Once these are established, optimisations can then be run on the system to calculate the optimal solution. These existing solutions are outlined in Section \ref{section1.1}.

\subsection{Simplifying This Model}
Similar to the techniques outlined in Section \ref{optimisationinwater}, the proposed system will include Pipes and Nodes. Pipes will be simplified to just connections with water flowing through them without consideration for pressure limits of pipes due to the scope of the project rendering this infeasible in the given time. Similarly, valves and storage tanks will also not be considered in this model due to the scope of this project.\newline
Moreover, a Manhattan grid will be enforced, this has several advantages:
\begin{enumerate}
    \item Simplicity: a regular, repetitive structure is easy to plan, model, and build — both conceptually and computationally.
    \item Ease of Navigation and Maintenance: predictable layouts make it easier for utility workers to locate pipelines, junctions, and issues in the field.
    \item Uniform Coverage: the grid naturally allows for even spacing between junctions or nodes, making it easier to guarantee access to every required point.
    \item Modularity: this system is scalable and extendable, therefore new areas can be added simply by expanding the grid.
    \item Simplifies Optimisation Formulation: the grid can be easily represented using coordinate systems, making it convenient for mathematical modelling.
\end{enumerate}
The inclusion of a limited number of regulatory constraints was considered for implementation within this application, however, due to the scope of this project and the many complexities associated with and surrounding regulatory compliance: it was deemed infeasible within the target time-frame and ultimately not included in the final project.\footnote{The project will retain the ability for these to be implemented as constraints upon the solver and can be considered for future work.}

%********************************** % Existing Software  **************************************
\section{Existing Software} 
\label{section1.1}

Presently, several different technologies are used to solve similar problems however none within the breadth of the author's research for this exact purpose. For instance, many applications exist for the design and construction of water piping within buildings but fail to consider a more macro-scale such as neighbourhood or area planning. As such, this section will examine and evaluate different applications and/or approaches that exist within the domain of piping network design.

\subsection{Aspen OptiRouter}\label{aspenoptirouter}

Aspen OptiRouter\textsuperscript{TM} is a 3D piping auto-routing application developed by AspenTech\textsuperscript{TM}  \citep{aspentech_2025}. It is designed to optimise the layout of piping systems within industrial facilities, such as chemical plants and refineries. OptiRouter\textsuperscript{TM} is a component of the Aspen OptiPlant\textsuperscript{TM} suite. This application drastically reduces the time needed to design piping layouts by generating multiple viable routes automatically. It's well-suited for complex industrial estates where manual routing is costly and time-intensive.\newline
Although this application contains a lot of the functionality the proposed application will contain, there are multiple shortcomings. Firstly, licensing and implementation costs can be substantial, limiting accessibility for smaller firms or, for a use-case of the proposed application, the rebuilding of conflict zones where funds are scarce. Moreover, there is a steep learning curve which requires significant training. Crucially, this application does not function on the macro-scale for area-planning.\newline
However, there are positive aspects to takeaway from this application: automated routing, cost estimation support, and the real-world applications will all be crucial for the proposed application.

\subsection{Modular Plant Design System 4 (MPDS4) Piping Design}

MPDS4 Piping Design is a software module used to design pipe routes in 3D. This application is once again specifically tailored for the design and layout of complex piping systems in industrial plants and factories. This software allows users to design pipe routes in 3D, supporting various standards and pipe specifications. Similarly to the application referenced in Section \ref{aspenoptirouter}, MPDS4 includes a rule-based automatic routing engine that automatically calculates feasible pipe paths between connection points. The application takes into account component sizes, available space, and layout rules. Following this, the application minimizes routing conflicts and reduces manual layout effort and allows for manual adjustment after auto-routing, providing design flexibility. This functionality is particularly valuable in industrial piping contexts, however, this application is primarily at the building or facility scale, not neighbourhood-scale utility planning; which the proposed application uniquely addresses. Although, manual adjustment would be a good feature to include, it is not feasible within the scope of this dissertation.

%********************************** % Approaches to the Problem  *************************************
\section{Approaches to the Problem}\label{problemapproaches}
Designing an efficient water distribution network is inherently an optimisation problem; this is due to a variety of factors. Firstly, the objective is to minimise the total cost of the network. This objective defines a quantitative goal to be minimised: a core feature of optimisation problems. The system must also determine which nodes to connect - a binary decision - and how much flow goes through each connection - a continuous variable. The design must also satisfy a range of constraints: flow conservation, capacity limits, and any regulatory guidelines the author imposes. These constraints restrict the solution space and must be enforced - another staple of optimisation. Furthermore, every design involves trade-offs; connecting more directly can save pipe length but might not meet flow requirements or reducing cost might increase system complexity or pressure loss. These trade-offs are best managed through optimisation, where the model can explore many configurations and select the one with the best balance. This structure forms a Mixed-Integer Programming problem, where both integer and continuous variables are optimized under linear constraints - a standard format for complex planning problems.

\subsection{NP-Completeness}
Determining NP-Completeness is important for the given problem as it determines what class of solver to use. A problem is NP-Complete if it is in NP\footnote{NP stands for Nondeterministic Polynomial Time. It refers to a class of decision problems where, for any given solution, it can be quickly verified in polynomial time if the solution is correct. \citep{computationalcomplexity}} and it is NP-Hard\footnote{A problem is NP-Hard if it is at least as hard as any NP problem. \citep{computationalcomplexity}}. The problem addressed is in NP as once a solution is proposed it can be verified efficiently, i.e. it is simple to check if all demands are met. It is NP-Hard as it includes characteristics of other NP-Complete problems such as the Steiner Tree Problem\footnote{The Steiner tree problem in graphs requires a tree of minimum weight that contains all nodes and minimizes the total weight of its edges. \citep{Robins2008}}.\newline
The fact of NP-Completeness now implies that no known algorithm can solve all instances effectively i.e. in polynomial time. As more variables are added to the problem, the computational difficulty increases rapidly. This reinforces the usage of Mixed Integer Programming (MIP) and, in particular, a MIP solver. Despite the problem's intractability, a MIP solver can efficiently solve the proposed problem using advanced solver techniques.

%********************************** % Evaluation of Solvers  *************************************
\section{Evaluation of Solvers}\label{evaluationofsolvers}
Although not the central focus of this paper, as part of the technological solution proposed it is necessary to integrate a reliable and efficient method for solving the underlying mathematical problems. In particular the routing of water through a network while satisfying capacity, demand, and pressure constraints. When doing this, it is important to consider the most suitable MIP solver for the task.\newline

For the scope of this paper, two leading commercial solvers are considered - Gurobi and IBM CPLEX. These are considered for their established performance, strong industry adoption, and ease of use - which can vary depending on the specific problem structure and other factors.\newline
This brief comparison highlights practical differences between Gurobi and CPLEX in the context of application development, focusing on factors such as ease of installation, API clarity, available documentation, and overall for integrating the optimisation components into a larger system.

\subsection{IBM CPLEX}
IBM CPLEX \citep{ibm_cplex} is an optimisation solver developed by IBM, capable of solving linear programming, quadratic programming, and, most importantly, MIP problems, as well as other mathematical models. It is widely adopted in both academia and industry for large-scale complex optimisation problem due to its robust framework and efficient solver.\newline
In the context of this project, as established in Section \ref{problemapproaches}, the core problem will be formulated as a MIP problem. CPLEX is well-suited for this use case, however, despite it's technical merits, there are several practical limitations associated with adopting CPLEX.\newline
CPLEX is a commercial software package: while an academic license is available, it requires separate approval and setup. While this is not critical, pricing for non-academic use is unlisted however is reported to range from \$15,000USD for a single user up to \$100,000USD for a full commercial rollout. This makes it incredibly difficult to justify its usage within the proposed project as this makes the final tool less accessible for organisations or users who might use, adapt, or integrate this project into their workflows. The proposed project will also be integrated in Java, due to the author's expertise in the language. While CPLEX does support Java through its Concert API, integrating and maintaining the CPLEX environment can be more difficult compared to other solvers that offer simpler Java usability. Given these considerations, IBM CPLEX will not be used for this project.
\subsection{Gurobi}
Similarly to IBM CPLEX, Gurobi \citep{gurobi} is an optimisation solver for mathematical programming: recognized for its performance in solving MIP problems and related optimisation problems. This framework has become an industry standard due to its speed, flexibility, and ease of use. Once again, as established in Section \ref{problemapproaches}, this is an MIP problem due to the involvement of discrete decisions and continuous variables: making it well-aligned with Gurobi's capabilities.\newline
Several factors support the usage of Gurobi for the proposed project. Gurobi consistently ranks as the fastest solvers in industry benchmarks, particularly for large MIP problems; consistently higher than IBM CPLEX. Similar to CPLEX, Gurobi also offers a fully featured academic license free of charge. There is a commercial license which, like IBM CPLEX, the price of which can vary as it is unlisted, however, is reported to be cheaper than IBM CPLEX. However, as with all software applications with adjustable prices, this can vary. Finally, Gurobi provides well-documented and actively maintained Java bindings, which align directly with the architecture of this project. This allows for seamless integration with the Java-based front-end and back-end components of the system without the need for additional wrappers or language bridges.\newline
Considering these advantages, Gurobi will be the optimisation solver for this project. It meets both the technical and practical requirements of the proposed project and provides a sustainable and high-performance foundation for future extensions or real-world adaptations of the tool.


\section{System Requirements}\label{systemrequirements}
From the outlined research, the following requirements have been defined:

\subsection{Functional Requirements}

\begin{longtable}{l p{13cm}}
\caption{Functional Requirements}\label{table:functionalrequirements} \\
\toprule
ID & Requirement \\
\midrule
\endfirsthead

\multicolumn{2}{c}{{\bfseries \tablename\ \thetable{} -- continued from previous page}} \\
\toprule
ID & Requirement \\
\midrule
\endhead

\midrule \multicolumn{2}{r}{{Continued on next page}} \\
\endfoot

\bottomrule
\endlastfoot

\multicolumn{2}{c}{\textbf{Node Placement}} \\
FR1 & The application shall allow the user to place nodes where they wish \\
FR2 & The type of node placed shall be able to be determined at the time of placement \\

\midrule
\multicolumn{2}{c}{\textbf{Node Data Management}} \\
FR3 & The application shall store all placed nodes in memory, potentially in a list of all nodes, along with the necessary data for each node \\
FR4 & Each building node shall store a required flow demand value \\
FR5 & Each source node shall store a maximum output supply value \\

\midrule
\multicolumn{2}{c}{\textbf{Optimisation}} \\
FR6 & The application shall include a button to start the optimisation process \\
FR7 & The optimisation shall determine a valid piping configuration from sources to buildings, minimising total costs under given constraints \\

\midrule
\multicolumn{2}{c}{\textbf{Optimisation Constraints}} \\
FR8 & The optimisation shall enforce relevant network constraints, such as pipe capacity and pressure limits \\

\midrule
\multicolumn{2}{c}{\textbf{Flow Network Modelling}} \\
FR9 & Pipes shall only connect nodes that are parallel with one another \\
FR10 & Each pipe shall be unidirectional — no bidirectional flow \\
FR11 & The optimisation model shall satisfy flow conservation at all nodes and enforce demand/supply constraints \\

\midrule
\multicolumn{2}{c}{\textbf{Visual Feedback}} \\
FR12 & After optimisation, the canvas shall visually display the selected piping connections using lines between connected nodes \\
FR13 & The UI shall be automatically refreshed as optimisation results are available \\
\end{longtable}


\subsection{Non-Functional Requirements}

\begin{longtable}{l p{13cm}}
\caption{Non-Functional Requirements}\label{table:nonfunctionalrequirements} \\
\toprule
ID & Requirement \\
\midrule
\endfirsthead

\multicolumn{2}{c}{{\bfseries \tablename\ \thetable{} -- continued from previous page}} \\
\toprule
ID & Requirement \\
\midrule
\endhead

\midrule \multicolumn{2}{r}{{Continued on next page}} \\
\endfoot

\bottomrule
\endlastfoot

\multicolumn{2}{c}{\textbf{Usability}}\\
NFR1 & It should be easy to differentiate between different node types, either by colour or a label\\
NFR2 & After optimisation is complete, the application should draw the connections in a distinct colour to distinguish them from node indicators\\
NFR3 & The node selection interface should be easy to use and responsive\\
\midrule
\multicolumn{2}{c}{\textbf{Performance}}\\
NFR4 & The optimisation process shall complete within a reasonable time for small to medium-sized networks\\
NFR5 & UI Interactions, for example: node placement, shall have $\leq$ 100ms latency to ensure responsiveness\\
\midrule
\multicolumn{2}{c}{\textbf{Scalability}}\\
NFR6 & The application shall support increasing number of nodes without significant UI performance degradation\\
NFR7 & Optimisation logic shall be designed to handle any given connectivity layout and large problem sizes with minimal bottlenecks\\
\midrule
\multicolumn{2}{c}{\textbf{Reliability}}\\
NFR8 & The application shall not crash or hang during UI interaction or while optimisation is in progress\\
NFR9 & The optimisation thread shall run independently from the UI to prevent freezing or blocking the interface\\
\midrule
\multicolumn{2}{c}{\textbf{Maintainability}}\\
NFR10 & The code shall be modular and separated into logical components\\
\midrule
\multicolumn{2}{c}{\textbf{Portability}}\\
NFR11 & The application shall run on any Java-compatible OS with Gurobi installed and licensed\\
\end{longtable}


\subsection{Optional Requirements}\label{optionalrequirements}

\begin{longtable}{l p{13cm}}
\caption{Optional Requirements}\label{table:optionalrequirements} \\
\toprule
ID & Requirement \\
\midrule
\endfirsthead

\multicolumn{2}{c}{{\bfseries \tablename\ \thetable{} -- continued from previous page}} \\
\toprule
ID & Requirement \\
\midrule
\endhead

\midrule \multicolumn{2}{r}{{Continued on next page}} \\
\endfoot

\bottomrule
\endlastfoot

\multicolumn{2}{c}{\textbf{The application may allow customisation of, but not limited to, the following parameters:}}\\
OFR1.1 & Pipe cost per unit of distance\\
OFR1.2 & Building Pressure Requirements\\
OFR1.3 & Source Pressure Output\\
\end{longtable}


%********************************** % Fourth Section  *************************************
\section{Approach to Development}\label{approachtodev}
The proposed application will be developed using principles of agile development. Agile development is a methodology which emphasises adaptability and iterative improvements. The project is divided into small, manageable functions such as GUI design, Gurobi Integration, and Gurobi constraint development - a key principle of agile development. Early, basic prototypes of the optimisation model will be developed first to verify functionality before creation of a wider application. As understanding of the optimisation model evolves, new constraints and performance considerations will be integrated. Finally, each prototype will be tested through use, this includes trial optimisations which allows bugs to be caught early and system behaviour to remain predictable. The flexibility of agile development compliments the research-driven nature of the project, where precise requirements will not always be known in advance and have to be derived through experimentation and review of the associated literature.